%%%%%%%%%%%%%%%%%%%%%%%%%%%%%%%%%%%%%%%%%%%%%%%%%%%%%%%%%%%%%%%%%%%%%%%%%%%%%%
% This is INSTLAT2E.STY, a file containing instructions for AUTHORS of
%  JOURNAL OF LIE THEORY
% for LaTeX2e files.
%
% $Id: instlat2e.tex 26 2007-11-06 18:54:36Z mhorn $
% ---------------------------------------------------------------------------
% Load the article class. The following options are supported for switching the
% document language:
%  english, french, german
%
% Using one of these options ensures that the babel package is loaded with
% the corresponding language, for correct hyphenations. 
% Also, theorem names etc. are adjusted.
%%%%%%%%%%%%%%%%%%%%%%%%%%%%%%%%%%%%%%%%%%%%%%%%%%%%%%%%%%%%%%%%%%%%%%%%%%%%%%


\documentclass{artjlt}

\usepackage{amsmath,amsfonts,amssymb}
\usepackage{amsmath, amsthm, amssymb, calrsfs, wasysym, verbatim, bbm, color, cite, graphics, geometry, cite}
\usepackage{hyperref}

\newtheorem{thm}{Theorem}
\newtheorem{defn}{Definition}
\newtheorem{conv}{Convention}
\newtheorem{rem}{Remark}
\newtheorem{lem}{Lemma}
\newtheorem{cor}{Corollary}

%%%%%%%%%%%%%%%%%%%%%%%%%%%%%%%%%%%%%%%%%%%%%%%%%%


\newcommand{\?}{\textbackslash}
\newcommand{\C}{\mathbb{C}}
\newcommand{\R}{\mathbb{R}}
\newcommand{\Z}{\mathbb{Z}}
\newcommand{\Endo}{\operatorname{End}}
\def\RR{\mathbf{R}}

\title{A Note on the Fractional Witt Algebra}
\author{[Drew Remmenga drewremmenga@gmail.com]}

\keywords{Infinite-dimensional Lie algebras, Generalized Witt Algebras, Virasoro Algebra, Zeta functions, Weyl Derivative}
\msc{17B68,16S32,17B66,17B40,81R10,11M41, 26A33, 17B68}
\begin{document}

\maketitle
\begin{abstract}
  This note extends the construction of the fractional Witt algebra from real to complex parameters. By leveraging the holomorphic Weyl derivative, which admits a natural complex extension via its action on Fourier modes, we generalize the structure constants of the algebra to complex-valued functions. Specifically, we show that the parameter $a$ in the generators $L_n^a = -ie^{-ia(n+1)\theta} \partial^a$ and the Gamma function factors $\Gamma_p(s)$ can be taken as complex numbers, while the indices $p, q$ remain integers. The resulting Lie bracket takes the form $[L_n^a, L_m^a] = A_{m,n}(s) \otimes L_{n+m}^a$, where $A_{m,n}(s)$ is now a complex-valued function. This complexification enriches the algebraic structure and opens new avenues for representation theory and applications in conformal field theory and integrable systems.
 \end{abstract}
\section{Introduction}
The Witt algebra is a fundamental object in the theory of infinite-dimensional Lie algebras, with deep connections to conformal field theory, integrable systems, and string theory. It is defined as the complex Lie algebra spanned by generators $\{L_n : n \in \mathbb{Z}\}$ satisfying the commutation relations
\[
[L_n, L_m] = (n - m)L_{n+m}.
\]
In recent years, there has been growing interest in fractional or deformed versions of this algebra, motivated by problems in theoretical physics and number theory. In particular, La Nave and Phillips \cite{LaNave2019} introduced a fractional Witt algebra parameterized by a real number $a$, using the Weyl fractional derivative. Their construction yields generators $L_n^a$ and structure constants involving Gamma functions. 

In this note, we observe that both the Weyl derivative and the Gamma function factors admit natural complex extensions. By allowing the parameter $a$ and the Gamma function arguments to take complex values, we obtain a complex-parameter Witt algebra with complex-valued structure constants. This generalization preserves the Lie algebra structure while significantly broadening its scope. Our approach builds on the holomorphic nature of the Weyl derivative and the analytic properties of the Gamma function, leading to a more flexible framework for further study.

\section{Witt Algebra}
The Witt Algebra is defined as follows \cite{Schottenloher1997} \cite{Nam1999}:
\begin{align*}
W &:= \C \{L_n: n \in \Z\} \\
L_n &:= -i e^{-i n \theta} \frac{d}{d \theta}
\end{align*}
Acting on Fourier representable functions:
\begin{align*}
f(\theta) &\in C^\infty (\mathbb{S},\C) \\
f(\theta) &=\sum_{n=-\infty}^{\infty} c_n e^{i n \theta}
\end{align*} 
So our Lie bracket is:
\begin{align*}
  [L_n, L_m] f &= L_n L_m f - L_m L_n f \\
  &= ((1-m)-(1-n))(- e^{-i n \theta - i m \theta}) \frac{d}{d \theta} f(\theta) \\
  &=(n-m)L_{n+m} f(\theta)
\end{align*}
\section{The Weyl Derivative}
The (holomorphic) Weyl derivative $\partial^s$ \cite{article} acts on Fourier representable functions 
$f(\theta)= \sum_{n=-\infty}^\infty a_n e^{i n \theta}, a_0 = 0$\footnote{In \cite{article} they define the zero mode to be zero to avoid dividing by zero.} by:
We can include constant zero modes in $\theta$ by mapping them to zero under strictly negative $s$.
\begin{align*}
  \partial^s f(\theta)= \sum_{n=-\infty}^\infty (in)^s a_n e^{i n \theta}
\end{align*}
Then we have:
\begin{align*}
  \sum_{n=-\infty}^\infty (in)^s a_n e^{i n \theta} &= \sum_{n=1}^\infty (in)^s a_n e^{i n \theta} + (-in)^s a_{-n} e^{-i n \theta} \\
  &= i^s \sum_{n=1}^\infty (n)^s a_n e^{i n \theta} + (-n)^s a_{-n} e^{-i n \theta} \\
\end{align*}
It can be shown these derivatives commute:
\begin{align*}
  \partial^\nu \partial^\mu f &= \partial^\nu i^\mu \sum_{n=1}^\infty (n)^\mu a_n e^{i n \theta} + (-n)^\mu a_{-n} e^{-i n \theta}
  &= \sum_{n=1}^\infty (n)^{\nu+\mu} a_n e^{i n \theta} + (-n)^{\mu+\nu} a_{-n} e^{-i n \theta}
\end{align*}
\qedsymbol
It is clear that $s$ can take on any value in $\C$. 
\section{The Complex Parameter Witt Algebra}
In \cite{La_Nave_2019} they build a fractional Witt Algebra:
\begin{align*}
  L_n^a := -i e^{-i a(n+1)\theta}\partial^a , a \in \R \\
  \Gamma_p (s) := \frac{\Gamma(a(s+p)+1)}{\Gamma(a(s+p-1)+1)} , p \in \Z\\
  A_{p,q} &:= \Gamma_p(s) - \Gamma_q(s) 
\end{align*}
Where $\Gamma$ is the Gamma function. 
With relations:
\begin{align*}
  [L_n^a,L_m^a] &= A_{m,n} (s) \otimes L_{n+m}^a
\end{align*}
We observe that $\Gamma(z) $ is defined for all complex $z \not \in - \mathbb{N}$. Furthermore in Section 3 the Weyl derivative can be taken from $\C$. Consequently the parameter $a$ in our generators can also be generalized to $\C$.
Therefore we have the extension of this algebra from $a \in \R$ to $a \in \C$ and $\Gamma \in \R$ to $\Gamma \in \C$ and therefore of $\Gamma_p \in \R$ to $\Gamma_p \in \C$ and lastly $A_{p,q} (s) \in \R$ to a value in $\C$. 
However, $p,q$ remain $\in \Z$. Consequently the structure constraints in $A$ become meromorphic functions, defining a Lie algebra over a field of complex-valued functions.  
\section{Conclusion}
We have shown that the fractional Witt algebra introduced in \cite{LaNave2019} admits a natural extension to complex parameters. By complexifying the Weyl derivative and the associated Gamma function factors, we obtain a family of Lie algebras parameterized by complex numbers, with structure constants given by complex-valued functions. This extension preserves the algebraic relations while enabling new connections to complex analysis, representation theory, and mathematical physics. 

Future work may include the study of representations of this complex-parameter algebra, its central extensions (e.g., a complex-parameter Virasoro algebra), and its applications in conformal field theory and integrable systems. The complexified structure also invites exploration of analytic properties of the structure constants, such as their behavior under meromorphic continuation and their relation to special functions.
\section*{Conflict of Interest}
On behalf of all authors, the corresponding author states that there is no conflict of interest.
\section*{Data Availability Statement}
This manuscript contains no external data libraries.
\bibliographystyle{plain}  % or another style like alpha, unsrt, etc.
\bibliography{references1.bib}  % the name of the .bib file
\end{document}


