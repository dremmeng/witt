%%%%%%%%%%%%%%%%%%%%%%%%%%%%%%%%%%%%%%%%%%%%%%%%%%%%%%%%%%%%%%%%%%%%%%%%%%%%%%
% This is INSTLAT2E.STY, a file containing instructions for AUTHORS of
%  JOURNAL OF LIE THEORY
% for LaTeX2e files.
%
% $Id: instlat2e.tex 26 2007-11-06 18:54:36Z mhorn $
% ---------------------------------------------------------------------------
% Load the article class. The following options are supported for switching the
% document language:
%  english, french, german
%
% Using one of these options ensures that the babel package is loaded with
% the corresponding language, for correct hyphenations. 
% Also, theorem names etc. are adjusted.
%%%%%%%%%%%%%%%%%%%%%%%%%%%%%%%%%%%%%%%%%%%%%%%%%%%%%%%%%%%%%%%%%%%%%%%%%%%%%%


\documentclass{artjlt}

\usepackage{amsmath,amsfonts,amssymb}
\usepackage{amsmath, amsthm, amssymb, calrsfs, wasysym, verbatim, bbm, color, cite, graphics, geometry, cite}
\usepackage{hyperref}

\newtheorem{thm}{Theorem}
\newtheorem{defn}{Definition}
\newtheorem{conv}{Convention}
\newtheorem{rem}{Remark}
\newtheorem{lem}{Lemma}
\newtheorem{cor}{Corollary}

%%%%%%%%%%%%%%%%%%%%%%%%%%%%%%%%%%%%%%%%%%%%%%%%%%


\newcommand{\?}{\textbackslash}
\newcommand{\C}{\mathbb{C}}
\newcommand{\N}{\mathbb{N}}
\newcommand{\R}{\mathbb{R}}
\newcommand{\Z}{\mathbb{Z}}
\newcommand{\Li}{\text{Li}}
\newcommand{\Endo}{\operatorname{End}}
\def\RR{\mathbf{R}}

\title{Complex Deformations of the Witt Algebra via Weyl and $q$-Calculus}
\author{[Drew Remmenga drewremmenga@gmail.com]}

\keywords{Infinite-dimensional Lie algebras, Generalized Witt Algebras, Virasoro Algebra, Zeta functions, Weyl Derivative}
\msc{17B68,16S32,17B66,17B40,81R10,11M41, 26A33, 17B68}
\begin{document}

\maketitle
\begin{abstract}
  This paper explores extensions of the Witt algebra to fractional and complex parameters using Weyl derivatives and $q$-calculus. 
  We generalize the classical Witt algebra by introducing complex-valued generators and structure functions, and we establish an analytic continuation of the $q$-Pochhammer symbol and the commutativity of complex parameter Weyl derivatives. 
  A fractional Leibniz rule is extended to complex parameters, and a complexified $q$-Witt algebra is constructed. 
  The results unify and extend earlier work by La Nave-Phillips and Purohit, providing a framework for infinite-dimensional Lie algebras with complex structure constants.
 \end{abstract}
\section{Introduction}
The Witt algebra is a fundamental object in the theory of infinite-dimensional Lie algebras, with deep connections to conformal field theory, integrable systems, and mathematical physics. Classically, it is defined over the complex numbers with generators $\{L_n : n \in \mathbb{Z}\}$ satisfying the bracket relation
\[
[L_n, L_m] = (n - m)L_{n+m}.
\]
In recent years, there has been growing interest in extending such structures to fractional and complex parameters, motivated by connections to $q$-analysis, fractional calculus, and meromorphic deformation theory.

In this paper, we build on the work of La Nave and Phillips \cite{La_Nave_2019}, who constructed a fractional Witt algebra with real parameters, and Purohit \cite{Leibniz} who derived a $q$-analogue of the Leibniz rule. By combining these approaches, we extend the generators and structure functions to the complex domain, allowing for a richer class of Lie algebras with complex-valued structure constants.
We outline continuous combinatorial functions for use in $q$-deformations. We extend the Weyl-derivative to complex parameters in $s$. Then we examine a complexification of the derivative parameter on three separate Witt structures. 

Our main contributions include:
\begin{itemize}
    \item A generalization of the $q$-Pochhammer symbol and its connection to the dilogarithm and $q$-Gamma function.
    \item An extension of the Weyl derivative to complex orders and a proof of its commutativity.
    \item A complexified fractional Leibniz rule and its application to $q$-deformed Witt algebras.
    \item The construction of three complex-parameter Witt algebras with meromorphic structure functions.
\end{itemize}

This work lays the groundwork for further study of deformed Virasoro algebras and their representations in the complex setting.
\section{Witt Algebra}
The Witt Algebra is defined as follows \cite{Schottenloher1997} \cite{Nam1999}:
\begin{align*}
W &:= \C \{L_n: n \in \Z\} \\
L_n &:= -i e^{-i n \theta} \frac{d}{d \theta}
\end{align*}
Acting on Fourier representable functions:
\begin{align*}
f(\theta) &\in C^\infty (\mathbb{S},\C) \\
f(\theta) &=\sum_{n=-\infty}^{\infty} c_n e^{i n \theta}
\end{align*} 
So our Lie bracket is:
\begin{align*}
  [L_n, L_m] f &= L_n L_m f - L_m L_n f \\
  &= ((1-m)-(1-n))(- e^{-i n \theta - i m \theta}) \frac{d}{d \theta} f(\theta) \\
  &=(n-m)L_{n+m} f(\theta)
\end{align*}
Or equivalently:
\begin{align*}
W &:= \C \{L_n: n \in \Z\} \\
L_n &:= -z^{n+1} \frac{d}{d z}
\end{align*}
Acting again on Fourier representable functions:
\begin{align*}
f(z)  &=\sum_{n=0}^{\infty} c_n z \\
z &\in \C
\end{align*} 
With the same bracket.
\section{The Combinatorics of $(a;q)_z$}
In \cite{díaz2024continuousanaloguespochhammersymbol} they extend the $k$-Pochhammer symbol from $R \times R \times \N \to R$ to a more general formulation:
\begin{align*}
  (x,y,z) &= y^z \frac{\Gamma(\frac{x}{y}+z)}{\Gamma(\frac{x}{y})}
\end{align*}
With this we derive a more general $q$-Pochhammer expression which is necessary for extending the $q$-derivative to complex values in $q$. 
\begin{align*}
  (a;q)_z = \frac{(a;q)_\infty}{(aq^z;q)_\infty} 
\end{align*}
By noting that: $\Gamma_q = \frac{(q;q)_\infty}{(q^z;q)_\infty} (1-q)^{1-z}$ we obtain:
\begin{align*}
  (a;q)_z &= \exp(-\frac{\Li_2 (aq^z)-\Li_2(a)}{\ln(q)})
\end{align*}
Where $\Li_2$ is the dilogarithm. Or equivalently:
\begin{align*}
  (a;q)_z &= \exp(\int_0^z\ln(1-aq^{t})dt)
\end{align*}\qedsymbol
\section{The Weyl and $q$ Derivatives}
\subsection{Weyl Derivatives Commute}
The (holomorphic) Weyl derivative $\partial^s$ \cite{article} acts on Fourier representable functions 
$f(\theta)= \sum_{n=-\infty}^\infty a_n e^{i n \theta}, a_0 = 0$\footnote{In \cite{article} they define the zero mode to be zero to avoid dividing by zero.} by:
\begin{align*}
  \partial^s f(\theta)= \sum_{n=-\infty}^\infty (in)^s a_n e^{i n \theta}
\end{align*}
Then we have:
\begin{align*}
  \sum_{n=-\infty}^\infty (in)^s a_n e^{i n \theta} &= \sum_{n=1}^\infty (in)^s a_n e^{i n \theta} + (-in)^s a_{-n} e^{-i n \theta} \\
  &= i^s \sum_{n=1}^\infty (n)^s a_n e^{i n \theta} + (-n)^s a_{-n} e^{-i n \theta} \\
\end{align*}
We can include constant zero modes in $\theta$ by mapping them to zero under strictly negative $s$.
It can be shown these derivatives commute:
\begin{align*}
  \partial^\nu \partial^\mu f &= \partial^\nu i^\mu \sum_{n=1}^\infty (n)^\mu a_n e^{i n \theta} + (-n)^\mu a_{-n} e^{-i n \theta}
  &= \sum_{n=1}^\infty (n)^{\nu+\mu} a_n e^{i n \theta} + (-n)^{\mu+\nu} a_{-n} e^{-i n \theta} \qedsymbol
\end{align*}
It is clear that $s$ can take on any value in $\C$. 
\subsection{Fractional Leibniz}
In \cite{Leibniz} they construct a Leibniz rule on an equivalent expression of Weyl derivatives:
\begin{align*}
  _zD^\alpha_{q,\infty} \{f(z)\} = \frac{q^{-\mu (1 +\mu)/2}}{\Gamma_q (- \mu)}\int_0^\infty (t -z)_{-\mu -1} f(t q^{1+\mu}) d(t;q)
\end{align*}
Where $\Gamma_q$ is once again the $q$-Gamma function. Where:
\begin{align*}
 \int_z^\infty f(t) d(t;q)=z(1-q)\sum_{k=1}^\infty q^{-k} f(zq^{-k})
\end{align*}
And in the particular case $f(z) = z^{-p}$ we have:
\begin{align*}
  _zD^\alpha_{q,\infty} \{z^{-p}\} = \frac{\Gamma_q(p+\alpha)}{\Gamma_q(p)}q^{- \alpha p+\alpha(1-\alpha)/2} z^{-p-\alpha} 
\end{align*}
They prove a fractional Leibniz Rule:
\begin{align*}
  _z D^\alpha_{q,\infty} \{U(z)V(z)\} &= \sum_{r=0}^\alpha \frac{(-1)^r q^{r(r+1)/2}(q^{-\alpha};q)_r}{(q;q)_r} \\
  & _z D^{\alpha-r}_{q,\infty} \{U(z)\}_z D^\alpha_{q,\infty} \{V(zq^{\alpha-r})\}
\end{align*}
By our work in Section 3 and Section 4.1 we can now extend to any complex $\alpha$ with a classic Riemann integral over $r$:
\begin{align}
  _z D^\alpha_{q,\infty} \{U(z)V(z)\} &= \int_0^\alpha \frac{(-1)^r q^{r(r+1)/2}(q^{-\alpha};q)_r}{(q;q)_r} \nonumber\\
  & _z D^{\alpha-r}_{q,\infty} \{U(z)\}_z D^\alpha_{q,\infty} \{V(zq^{\alpha-r})\} dr \nonumber\\
  A(r) & = \frac{(-1)^r q^{r(r+1)/2}(q^{-\alpha};q)_r}{(q;q)_r} \label{A(r)}
\end{align}
\section{The Complex Parameter Witt Algebras}
\subsection{La Nave's and Philips' Work}
In \cite{La_Nave_2019} they build a fractional Witt Algebra:
\begin{align*}
  L_n^{a'} &:= -i e^{-i a(n+1)\theta}\partial^a , a \in \R \\
  \Gamma_p (s) &:= \frac{\Gamma(a(s+p)+1)}{\Gamma(a(s+p-1)+1)} , p \in \Z\\
  A_{p,q} &:= \Gamma_p(s) - \Gamma_q(s) 
\end{align*}
Where $\Gamma$ is the Gamma function. 
We have bracket:
\begin{align*}
  [L_n^{a'},L_m^{a'}] &= A_{m,n} (s) \otimes L_{n+m}^{a'}
\end{align*}
We observe that $\Gamma(z) $ is defined for all complex $z \not \in - \mathbb{N}$. Furthermore in Section 4 the Weyl derivative can be taken from $\C$. Consequently the parameter $a$ in our generators can also be generalized to $\C$.
Therefore we have the extension of this algebra from $a \in \R$ to $a \in \C$ and $\Gamma(z) \in \R$ to $\Gamma(z) \in \C$ for a meromorphic structure constraint in $A$. And therefore of $\Gamma_p \in \R$ to $\Gamma_p \in \C$ and lastly $A_{p,q} (s) \in \R$ to a value in $\C$. 
However, $p,q$ $(n,m)$ remain $\in \Z$. Consequently the structure constraints in $A$ become meromorphic functions, defining a Lie algebra over a field of complex-valued functions.  
\qedsymbol
\subsection{Complexified q-Witt Relations in Purohit's Construction}
We define:
\begin{align*}
  L''^\alpha_n &:= -z^{n+1} _zD_{q,\infty}^\alpha, \alpha \in \C, q \in \C \\
  W'' &:= \C\{L^\alpha_n, n \in \Z\}
\end{align*}
We now compute the bracket on an arbitrary $V(z)$. 
\begin{align}
  V(z) &= \sum_{k=0}^\infty z^k C_k \nonumber \\
  _zD_{q,\infty}^\alpha V(z) &= \sum_{k=0}^\infty C_k \frac{\Gamma_q(-k + \alpha)}{-k} q^{\alpha k + \alpha(1-\alpha)/2} z^{k-\alpha}\nonumber \\
  _zD_{q,\infty}^\alpha V(zq^{\alpha-r}) &= \sum_{k=0}^\infty C_k \frac{\Gamma_q(-k + \alpha)}{\Gamma_q(-k)} q^{\alpha k + \alpha(1-\alpha)/2} z^{k-\alpha} q^{(k-\alpha)(\alpha -r)}\nonumber \\
  (_zD_{q,\infty}^\alpha) _zD_{q,\infty}^\alpha V(zq^{\alpha -r})&= \sum_{k=0}^\infty C_k \frac{\Gamma_q(-k + 2\alpha)}{\Gamma_q(-k+\alpha)}  \frac{\Gamma_q(-k + \alpha)}{\Gamma_q(-k)} \nonumber\\
  &q^{\alpha k + \alpha(1-\alpha)/2}q^{\alpha (k-\alpha) + \alpha(1-\alpha)/2} q^{(k-\alpha)(\alpha -r)}z^{k-2\alpha}  \nonumber \\
  g(r) &= \sum_{k=0}^\infty C_k  \frac{\Gamma_q(-k + 2\alpha)}{\Gamma_q(-k)} q^{3\alpha k -2\alpha^{2} + \alpha(1-\alpha)-rk+r \alpha} z^{k-2\alpha} \label{g(r)}
\end{align}
We need the $\alpha-r$ derivative on $-z^{m+1}$:
\begin{align}
  f(r,m) = _zD_{q,\infty}^{\alpha-r} (-z^{m+1}) &=\frac{\Gamma_q(-m -1+\alpha-r)}{\Gamma_q(-m -1)}q^{(\alpha-r)m+(\alpha-r)(1-\alpha+r)/2}z^{m+1 -\alpha +r} \label{f(r,m)}
\end{align}
We have a half-bracket:
\begin{align}
  L^{''\alpha}_{n} L^{''\alpha}_{m}V(z) &= \sum_{k=0}^{\infty} C_k \frac{\Gamma_q(-k+2\alpha)}{\Gamma_q(-k)} \cdot z^{k + m + 1 - 3\alpha} \cdot q^{3\alpha k + \alpha m - \frac{7\alpha^2}{2} + \frac{3\alpha}{2}} \nonumber\\
&\cdot \int_{0}^{\alpha} \frac{(-1)^{r}(q^{-\alpha};q)_{r}}{\Gamma_q(-m-1)(q;q)_{r}} \Gamma_q(-m-1+\alpha-r) \cdot q^{\,r\left(-\frac{r}{2} + 2\alpha - k - m - \frac{1}{2}\right)} z^{\,r}  dr \nonumber
\end{align}
So by combining \ref{f(r,m)}, \ref{g(r)}, and \ref{A(r)} have relations:
\begin{align*}
  [L^{''\alpha}_n , L^{''\alpha}_m] &= \int_0^\alpha A(r) g(r) (f(r,m)-f(r,n)) dr
\end{align*}
Clearly the bracket is anti-symmetric and satisfies the Jacobi identity.
\qedsymbol $\\$
\subsection{A Middle Ground Deformation}
We define:
\begin{align*}
  L'''^\alpha_n &:= -z^{n+1} \partial^\alpha, \alpha \in \C\\
  W''' &:= \C\{L'''^\alpha_n, n \in \Z\} \\
  f(z) &= \sum_{k=0}^\infty C_k z^k
\end{align*}
With:
\begin{align*}
  L_n^\alpha L_m^\alpha f(z) &= -z^{n+1} \partial^\alpha (-z^{m+1})\partial^\alpha \sum_{k=0}^\infty C_k z^k \\
  &= \sum_{k=0}^\infty-(k+m+1-\alpha)^\alpha z^{-\alpha} L_{n+m}^\alpha f(z) \\
\end{align*}
So we have relations:
\begin{align*}
  [L_n^\alpha, L_m^\alpha] f(z) &=   &= \sum_{k=0}^\infty \left((k+n+1-\alpha)^{\alpha} - (k+m+1-\alpha)^{\alpha} \right) z^{-\alpha} L_{n+m}^{\alpha}f(z)
\end{align*}
\qedsymbol $\\$
This isn't quite a Lie-Algebra structure as it depends on $k$ but it is a related algebra of deformed Fourier representations on the circle. Indeed, it is the algebra of Witt operators on the integers (a spectrum on $k \in \Z$). 
\section{Conclusion}
We have constructed several generalizations of the Witt algebra by introducing complex and fractional parameters into its defining structure. Using the Weyl derivative and $q$-calculus, we extended the classical Witt algebra to complex-valued generators and structure functions, leading to a family of Lie algebras whose brackets are governed by meromorphic functions. 
The most significant contributions are the complexifications in $W'$ and the examination of the relations in $W'''$. 
The complexification of the La Nave–Phillips and Purohit constructions allows for a unified treatment of fractional and $q$-deformed Witt algebras. Our results suggest natural directions for future research, including the study of central extensions, representation theory, and applications to conformal field theory and integrable systems with complex parameters.
\section*{Conflict of Interest}
On behalf of all authors, the corresponding author states that there is no conflict of interest.
\section*{Data Availability Statement}
This manuscript contains no external data libraries.
\bibliographystyle{plain}  % or another style like alpha, unsrt, etc.
\bibliography{references1.bib}  % the name of the .bib file
\end{document}


