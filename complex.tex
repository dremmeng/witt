%%%%%%%%%%%%%%%%%%%%%%%%%%%%%%%%%%%%%%%%%%%%%%%%%%%%%%%%%%%%%%%%%%%%%%%%%%%%%%
% This is INSTLAT2E.STY, a file containing instructions for AUTHORS of
%  JOURNAL OF LIE THEORY
% for LaTeX2e files.
%
% $Id: instlat2e.tex 26 2007-11-06 18:54:36Z mhorn $
% ---------------------------------------------------------------------------
% Load the article class. The following options are supported for switching the
% document language:
%  english, french, german
%
% Using one of these options ensures that the babel package is loaded with
% the corresponding language, for correct hyphenations. 
% Also, theorem names etc. are adjusted.
%%%%%%%%%%%%%%%%%%%%%%%%%%%%%%%%%%%%%%%%%%%%%%%%%%%%%%%%%%%%%%%%%%%%%%%%%%%%%%


\documentclass{artjlt}

\usepackage{amsmath,amsfonts,amssymb}
\usepackage{amsmath, amsthm, amssymb, calrsfs, wasysym, verbatim, bbm, color, cite, graphics, geometry, cite}
\usepackage{hyperref}

\newtheorem{thm}{Theorem}
\newtheorem{defn}{Definition}
\newtheorem{conv}{Convention}
\newtheorem{rem}{Remark}
\newtheorem{lem}{Lemma}
\newtheorem{cor}{Corollary}

%%%%%%%%%%%%%%%%%%%%%%%%%%%%%%%%%%%%%%%%%%%%%%%%%%


\newcommand{\?}{\textbackslash}
\newcommand{\C}{\mathbb{C}}
\newcommand{\R}{\mathbb{R}}
\newcommand{\Z}{\mathbb{Z}}
\newcommand{\Endo}{\operatorname{End}}
\def\RR{\mathbf{R}}
\keywords{Weyl Derivative, Complexification, Zeta Functions}
\msc{11M41,17B65,17B66,81T40}

\title{A Complex Parameter Extension of the Witt Algebra}
\author{[Drew Remmenga drewremmenga@gmail.com]}

\keywords{Infinite-dimensional Lie algebras, generalized Witt algebras, Virasoro algebra,zeta functions, Weyl Derivative}
\msc{17B68,16S32,17B66,17B40,81R10,11M41, 26A33, 17B68}
\begin{document}

\maketitle
\begin{abstract}
  We present a complexification of the Witt algebra by introducing a continuous complex parameter into its defining derivation. The standard Witt algebra W is the complex Lie algebra spanned by operators acting on the circle, with the well-known bracket $[L_n,L_m]= L_{n+m}$. 
  By replacing the ordinary derivative with the Weyl fractional derivative. We compute the Lie bracket of these operators and show that it closes, yielding a new infinite-dimensional "Complexified Witt Algebra." 
  This construction extends the symmetry of the classical Witt algebra and provides a new algebraic structure beyond the standard Virasoro framework.
  \end{abstract}
\section{Introduction}
Infinite-dimensional Lie algebras play a fundamental role in various branches of mathematics and theoretical physics, most notably in conformal field theory and string theory. Among these, the Witt algebra and its central extension, the Virasoro algebra, are of paramount importance. They describe the infinitesimal conformal transformations on the circle and underpin the algebraic structure of two-dimensional conformal field theories.\cite{Schottenloher1997}
The classical Witt algebra is generated by vector fields acting on the circle, which act on the space of smooth complex-valued functions. Recent research has explored generalizations of this structure, including deformations, extensions to higher dimensions, and the introduction of fractional operators inspired by fractional calculus \cite{pakianathan2010generalizedwittalgebrasvariable}\cite{La_Nave_2019}.
In this paper, we diverge from the path of constructing a fractional Virasoro algebra and instead propose a direct complexification of the Witt algebra. Our approach leverages the Weyl fractional derivative $\partial^\nu$, a tool that generalizes the standard derivative to a complex order $\nu$. By defining new generators $L^\nu_n:= -i e^{-i n \theta}\partial^\nu$, we embed the Witt algebra into a larger family of operators parameterized by both a discrete mode index $n\in \Z$
 and a continuous complex parameter $\nu \in \C$.
We demonstrate that the Lie bracket of these generalized operators yields a closed algebra $[L^\nu_n,L^\mu_m]=(n-m)L_{m+n}^{\nu+\mu}$, 
which we call the Complexified Witt Algebra. 
This result significantly extends the symmetry of the classical Witt algebra, introducing a continuous spectral parameter into the algebraic structure. The connection to generalized zeta functions through the action of the Weyl derivative suggests deep links to analytic number theory and spectral theory.
This paper is structured as follows: In Section 2, we review the standard Witt algebra. In Section 3, we introduce the Weyl derivative and its connection to zeta functions. Finally, in Section 4, we present the construction of the complexified Witt algebra and derive its fundamental commutation relations.
\section{Witt Algebra}
The Witt Algebra is defined as follows \cite{Schottenloher1997} \cite{Nam1999}:
\begin{align*}
W &:= \C \{L_n: n \in \Z\} \\
L_n &:= -i e^{-i n \theta} \frac{d}{d \theta}
\end{align*}
Acting on Fourier representable functions:
\begin{align*}
f(\theta) &\in C^\infty (\mathbb{S},\C) \\
f(\theta) &=\sum_{n=-\infty}^{\infty} c_n e^{i n \theta}
\end{align*} 
So we have a bracket:
\begin{align*}
  [L_n, L_m] f &= L_n L_m f - L_m L_n f \\
  &= ((1-m)-(1-n))(- e^{-i n \theta - i m \theta}) \frac{d}{d \theta} f(\theta) \\
  &=(n-m)L_{n+m} f(\theta)
\end{align*}
\section{The Weyl Derivative}
\subsection{Generalized Zeta Functions}
We need notation; We can generalize a zeta function in the numerator of each summnation term by:
\begin{align*}
  L(s,\chi(n)) = \sum_{n=1}^{\infty}\frac{\chi(n)}{n^s}
\end{align*}
Where $\chi(n)=a_0e^{in \theta}$ depends only on $n,\theta$ but could be complex. 
\subsection{Weyl Derivatives Commute}
The Weyl derivative $\partial^s$ \cite{article} acts on fourier representable functions 
$f(\theta)= \sum_{n=-\infty}^\infty a_n e^{i n \theta}, a_0 = 0$\footnote{In \cite{article} they define the zero mode to be zero to avoid dividing by zero.} by:
\begin{align*}
  \partial^s f(\theta)= \sum_{n=-\infty}^\infty (in)^s a_n e^{i n \theta}
\end{align*}
Then we have:
\begin{align*}
  \sum_{n=-\infty}^\infty (in)^s a_n e^{i n \theta} &= \sum_{n=1}^\infty (in)^s a_n e^{i n \theta} + (-in)^s a_{-n} e^{-i n \theta} \\
  &= i^s \sum_{n=1}^\infty (n)^s a_n e^{i n \theta} + (-n)^s a_{-n} e^{-i n \theta} \\
  \chi(n) &= a_n e^{i n \theta} \\
  &= i^s L(-s,\chi(n)) + i^s (-1)^s L(-s,\chi(-n)) \\
\end{align*}
It can be shown these derivative commute:
\begin{align*}
  \partial^\nu \partial^\mu f &= \partial^\nu i^\mu L(-\mu,\chi(n)+ i^\mu e^{i \pi \mu})L(-\mu,\chi(-n)) \\
  &= i^{\nu + \mu} L(-\mu-\nu,\chi(n)) + i^{\mu + \nu} e^{i \pi (\nu+\mu)} L(-\mu-\nu,\chi(-n))
\end{align*}
\section{Complexified Witt Algebra}
Diverging from the Fractional Virasoro algebra construction in \cite{La_Nave_2019}, we construct:
\begin{align*}
W &:= \C \{L^\nu_n: n \in \Z, \nu \in \C \} \\
L^\nu_n &:= -i e^{-i n \theta} \partial^\nu
\end{align*}
Where we can apply $\partial$ to $f$ with a nonzero $a_0$ by setting $\partial^\nu a_0 e^{0}:= 0, \forall \nu<0$ which feels justifified because the derivative of a constant is zero. This chosen rule extends the classic Weyl derivative to more functions $f$.  
Then we derive:
\begin{align*}
  [L^\nu_n, L^\mu_m] f &= L^\nu_n L^\mu_m f - L^\mu_m L^\nu_n f \\
  &= ((1-m)-(1-n))(- e^{-i n \theta - i m \theta}) \partial^{\nu+\mu} f(\theta) \\
  &=(n-m)L^{\nu + \mu}_{n+m} f(\theta)
\end{align*}
Then we have a complexified Witt symmetry.
\section{Conclusion}
In this paper, we have constructed a complex-parameter extension of the classical Witt algebra. By replacing the standard derivative in the generators with the Weyl fractional derivative $\partial^\nu$ of complex order $\nu$, we introduced a new family of operators $\{L^\nu_n\}$ parameterized by both a discrete mode index $n \in \mathbb{Z}$ and a continuous complex parameter $\nu \in \mathbb{C}$. We have demonstrated that the Lie bracket of these operators closes, yielding the commutation relation
\[
[L^\nu_n, L^\mu_m] = (n - m) L^{\nu + \mu}_{n + m}.
\]
This defines the \emph{Complexified Witt Algebra}, a significant generalization of the standard Witt algebra that incorporates a continuous spectral parameter into its fundamental structure.

This construction extends the symmetry of conformal transformations on the circle in a novel way. Unlike central extensions or fractional deformations that modify the form of the bracket, our approach preserves the Witt structure $(n-m)$ while augmenting the generators with an additive complex parameter. The requirement to carefully define the action on the constant term ($a_0 = 0$) is a natural consequence of working with the Weyl derivative on the circle.

Several intriguing directions for future research emerge from this work:
\begin{itemize}
    \item \textbf{Central Extension:} An immediate question is whether this algebra admits a non-trivial central extension, analogous to the Virasoro algebra, and what form the central term $c(\nu, \mu)$ would take. \footnote{The notation of $c$ as a special value of the classic Riemann Zeta function in many physics papers is suggestive in light of my own derivation of zeta functions}
    \item \textbf{Representation Theory:} The existence of the continuous parameter $\nu$ suggests the possibility of new classes of representations. Constructing and classifying these representations would be a fundamental next step.
    \item \textbf{Connection to Analytic Number Theory:} As hinted by the appearance of Dirichlet series in the action of the Weyl derivative, a deeper investigation into the relationship between this algebra and generalized zeta functions could unveil new connections between infinite-dimensional Lie algebras and analytic number theory.
    \item \textbf{Physical Applications:} The potential applications in conformal field theory and string theory are compelling. This complexified symmetry may describe new degrees of freedom or be relevant in theories with fractional or complex scaling dimensions.
\end{itemize}
In summary, the Complexified Witt Algebra provides a rich new algebraic structure that broadens our understanding of Witt-type symmetries and opens a promising avenue for cross-disciplinary exploration in mathematics and theoretical physics.
\section*{Conflict of Interest}
On behalf of all authors, the corresponding author states that there is no conflict of interest.
\section*{Data Availability Statement}
This manuscript contains no external data libraries.
\bibliographystyle{plain}  % or another style like alpha, unsrt, etc.
\bibliography{references1.bib}  % the name of the .bib file
\end{document}


