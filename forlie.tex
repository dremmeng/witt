%%%%%%%%%%%%%%%%%%%%%%%%%%%%%%%%%%%%%%%%%%%%%%%%%%%%%%%%%%%%%%%%%%%%%%%%%%%%%%
% This is INSTLAT2E.STY, a file containing instructions for AUTHORS of
%  JOURNAL OF LIE THEORY
% for LaTeX2e files.
%
% $Id: instlat2e.tex 26 2007-11-06 18:54:36Z mhorn $
% ---------------------------------------------------------------------------
% Load the article class. The following options are supported for switching the
% document language:
%  english, french, german
%
% Using one of these options ensures that the babel package is loaded with
% the corresponding language, for correct hyphenations. 
% Also, theorem names etc. are adjusted.
%%%%%%%%%%%%%%%%%%%%%%%%%%%%%%%%%%%%%%%%%%%%%%%%%%%%%%%%%%%%%%%%%%%%%%%%%%%%%%

\documentclass{artjlt}

\usepackage{amsmath,amsfonts,amssymb}
\usepackage{amsmath, amsthm, amssymb, calrsfs, wasysym, verbatim, bbm, color, cite, graphics, geometry, cite}
\usepackage{hyperref}

\newtheorem{thm}{Theorem}
\newtheorem{defn}{Definition}
\newtheorem{conv}{Convention}
\newtheorem{rem}{Remark}
\newtheorem{lem}{Lemma}
\newtheorem{cor}{Corollary}

%%%%%%%%%%%%%%%%%%%%%%%%%%%%%%%%%%%%%%%%%%%%%%%%%%

\renewcommand{\firstheadline}{\begin{minipage}[t]{9cm}
    Journal of * \hfill

    Volume {\bf 0}\ (2025)\ \hfill

    \copyright\  2025\ Drew Remmenga \hfill
    \end{minipage}\hfill
      }


\newcommand{\?}{\textbackslash}
\newcommand{\C}{\mathbb{C}}
\newcommand{\R}{\mathbb{R}}
\newcommand{\Endo}{\operatorname{End}}
\def\RR{\mathbf{R}}


\title{The Generalized Virasoro Algebra acting on $\C[\R]$}
\author{[Drew Remmenga drewremmenga@gmail.com]}
\keywords{Infinite-dimensional Lie algebras, generalized Witt algebras, Virasoro algebra, spectral invariants, self-centralizing, endomorphisms}
\msc{17B68,16S32,17B66,17B40,81R10}
\begin{document}
\nocite{*}

\maketitle
\begin{abstract}
I introduce and systematically analyze a family of infinite-dimensional Lie algebras constructed from the space $A$ which I define as the unital, associative, commutative algebra $A=\C[\R]$ which I define more formally in the introduction. 
I uncover four distinct, non-trivial Lie algebra structures on $A$ defined via the operations of formal differentiation, integration, logarithmic differentiation, and product integration. 
I prove that the algebra defined by integration is self-centralizing. Furthermore, I compute the explicit commutation relations for these algebras on a Fourier basis, demonstrating that the bracket defined by differentiation recovers a form of the centerless Virasoro algebra, thereby placing these constructions within the established framework of conformal field theory. This work provides a unified algebraic framework for generating and studying generalizations of the Witt and Virasoro algebras, revealing a rich structure of non-isomorphic examples from a single underlying space.
\end{abstract}
\section{Introduction}
   The Witt algebra, the Lie algebra of derivations on the circle, and its central extension, the Virasoro algebra, are foundational objects in mathematics and theoretical physics, particularly in conformal field theory (CFT). A natural and active area of research involves constructing and classifying generalizations of these algebras. 
   A common approach, explored in \cite{Dokovic1998} \cite{Kac1974} \cite{Kac1990} \cite{Nam1999} \cite{pakianathan2010generalizedwittalgebrasvariable}, is to define them as derivations on various commutative algebras. $\\$ $\\$
Let $A$ be a formal algebraic model inspired by the structure of tempered distributions. 
I define $A$ to be the commutative, associative, unital algebra over $\C$ with a basis $e^{i \pi \xi x} = \{e_\xi\}_{\xi \in \R}$ indexed by the real numbers where the multiplicative structure is defined on basis elements by:
\begin{align*}
   e_\xi \cdot e_\mu = e_{\xi + \mu}
\end{align*} 
And extended linearly on the specificed basis. The element $e_0$ serves as the multiplicative identity. This structure is isomorphic to the group algebra $\C[\R]$. $A$ is akin to a generalized Fourier basis \cite{Hormander1958}. $\\$ $\\$ 
The central question I address is: What natural Lie algebraic structures can be built directly on this space, and what are their properties? $\\$$\\$
Our main results provide a four-fold answer to this question:
\begin{itemize}
\item I prove that $A$ is closed under four fundamental operations: formal differentiation $(\partial)$, formal integration $(\int $with constant $C=0)$, logarithmic differentiation $(LD)$, and a defined product integration.
\item I demonstrate that each of these operatiuon induces a distinct Lie algebra structure on $A$ via a specific bracket, and I rigorously prove each bracket satisfies the Lie algebra axionms.
\item I establish a key structural property: the Lie algebra generated by the integration operator $Witt(\C[x,x^{-1}])$ with $\int$ is self centralizing. 
\item I compute the commutation relations on the given basis, showing that the bracket $[.,.]_\partial$ yields a Virasoro type algebra, thus connecting our abstract construction directly to the well-established formalism of CFT \cite{Lundholm2005} \cite{Schottenloher1997} \cite{teschner2017guidetwodimensionalconformalfield}.
\end{itemize}
This approach yields a rich family of both simple and non-simple generalized Witt algebras from a single construction. By leveraging formal calculus, pseudomonoid structures, and spectral ideas, I provide a unified framework for generating and analyzing these algebras. 
The results highlight the algebraic richness of $A$ and open new avenues for constructing representations and studying central extensions of these new Lie algebraic structures.
\section{Closure}
   Closure under the operations is built into the definition of $A$.
   \begin{thm} \label{thm:1}
      Closure of $A$ under $\partial$ (formal differentiation).
   \end{thm}
   \begin{proof}
      It is sufficient to discuss the operator on an arbitrary basis element. $\partial e_\xi \to 2 \pi i \xi e_{\xi} \in A$.
   \end{proof}
   \begin{thm} \label{thm:2}
      Closure of $A$ under $\int$ (formal antiderivative, $C:=0$). 
   \end{thm}
   \begin{proof}
      Take our $e_\xi \in A$ and send it to $\frac{1}{2 \pi i \xi} e_{\xi} \in A$. Note that when $\xi = 0, \implies e_0 = 1$. I know $A$ is closed under this operation by \cite{Hormander1958}. 
   \end{proof}
   \begin{thm}
      Closure of $A$ under logarithmic differentiation $(LD(f): f \to \frac{\partial f}{f})$ with $\text{LD}(f=0) := 0$.
   \end{thm}
   \begin{proof}
      I know $A$ is closed under division by \cite{Hormander1958} and Theorem \ref{thm:2} and under multiplication by construction. I know $A$ is closed $\partial$ by Theorem \ref{thm:1}. So if $f \in A, \frac{f'}{f} \in A$.
   \end{proof}
   \begin{thm} \label{thm:4}
      Closure of $A$ under product integration.
   \end{thm}
   \begin{proof}
      Take our $f$. Then by Theorem \ref{thm:2} $\int f = g \in A$. Then $\exp(g) \to \sum_{j=0}^{\infty} \frac{g^j}{j!} - 1 = F_f \in A$. \footnote{We define the product integral with the subtraction of one after the exponent so that the zero function $f=0$ doesn't 'move' under the operation.} 
\end{proof}
\section{Lie Algebra Structures}
   \begin{thm}
The space $(A, [\cdot,\cdot]_\partial)$ with differentiation forms a Lie algebra.
\end{thm}
\begin{proof}
I define the Lie bracket via derivative operators:
\[
[f,g]_\partial := f \cdot \partial g - g \cdot \partial f 
\]
where $\partial$ denotes the formal differentiation operator.

\noindent\textbf{Antisymmetry:}
\[
[g,f]_\partial = g \cdot \partial f - f \cdot \partial g = - (f \cdot \partial g - g \cdot \partial f) = -[f,g]_\partial
\]
Thus the bracket is antisymmetric:
\[
\boxed{[f,g]_\partial = -[g,f]_\partial} 
\]

\noindent\textbf{Jacobi Identity:}
I must verify:
\[
[f,[g,h]_\partial]_\partial + [g,[h,f]_\partial]_\partial + [h,[f,g]_\partial]_\partial = 0
\]
First expand one term:
\[
[f,[g,h]_\partial]_\partial = f \cdot \partial[g,h]_\partial - [g,h]_\partial \cdot \partial f
\]
where $[g,h]_\partial = g \cdot \partial h - h \cdot \partial g$. Applying $\partial$:
\[
\partial[g,h]_\partial = \partial(g \cdot \partial h) - \partial(h \cdot \partial g)
\]
Continuing:
\[
[f,[g,h]_\partial]_\partial = f \cdot (\partial(g \cdot \partial h) - \partial(h \cdot \partial g)) - (g \cdot \partial h - h \cdot \partial g) \cdot \partial f
\]

Similarly expanding the other two terms and summing:
\[
\begin{aligned}
&[f,[g,h]_\partial]_\partial + [g,[h,f]_\partial]_\partial + [h,[f,g]_\partial]_\partial \\
&= \big(f \partial(g \partial h) - f \partial(h \partial g) - g \partial h \partial f + h \partial g \partial f\big) \\
&\quad + \big(g \partial(h \partial f) - g \partial(f \partial h) - h \partial f \partial g + f \partial h \partial g\big) \\
&\quad + \big(h \partial(f \partial g) - h \partial(g \partial f) - f \partial g \partial h + g \partial f \partial h\big) \\
&= 0
\end{aligned}
\]
All terms cancel pairwise, proving:
\[
\boxed{[f,[g,h]_\partial]_\partial + [g,[h,f]_\partial]_\partial + [h,[f,g]_\partial]_\partial = 0}
\]

\footnote{Recall I am taking all of our integration constants $C$ to be zero}
\end{proof}
\begin{thm}
The space $(A, [\cdot,\cdot])$ with integration $\int$ forms a Lie algebra.
\end{thm}

\begin{proof}
I define the Lie bracket via integration operators:
\[
[f,g]\int := f \cdot \int g - g \cdot \int f 
\]
where $\int $ denotes the formal integration operator (anti-derivative).

\noindent\textbf{Antisymmetry:}
\[
[g,f]\int = g \cdot \int f - f \cdot \int g = - (f \cdot \int g - g \cdot \int f) = -[f,g]
\]
Thus the bracket is antisymmetric:
\[
\boxed{[f,g]\int = -[g,f]\int} 
\]
\noindent\textbf{Jacobi Identity:}
I must verify:
\[
[f,[g,h]\int]\int + [g,[h,f]\int]\int + [h,[f,g]\int]\int = 0
\]
First expand one term:
\[
[f,[g,h] \int]\int = f \cdot \int [g,h] \int - [g,h] \int \cdot \int f
\]
where $[g,h]\int = g \cdot \int h - h \cdot \int g$. Applying $\int $:
\[
\int [g,h] \int= \int (g \cdot \int h) - \int (h \cdot \int g)
\]
Continuing:
\[
[f,[g,h]\int]\int = f \cdot (\int (g \cdot \int h) - \int (h \cdot \int g)) - (g \cdot \int h - h \cdot \int g) \cdot \int f
\]

Similarly expanding the other two terms and summing:
\[
\begin{aligned}
&[f,[g,h]\int]\int + [g,[h,f]\int]\int + [h,[f,g]\int]\int \\
&= \big(f \int (g \int h) - f \int (h \int g) - g \int h \int f + h \int g \int f\big) \\
&\quad + \big(g \int (h \int f) - g \int (f \int h) - h \int f \int g + f \int h \int g\big) \\
&\quad + \big(h \int (f \int g) - h \int (g \int f) - f \int g \int h + g \int f \int h\big) \\
&= 0
\end{aligned}
\]
All terms cancel pairwise, proving:
\[
\boxed{[f,[g,h]\int]\int + [g,[h,f]\int]\int + [h,[f,g]\int]\int = 0} 
\]


\end{proof}
   
   \begin{thm}
The space $(A, [\cdot,\cdot]_{\text{LD}})$ with the logarithmic derivative bracket forms a Lie algebra.
\end{thm}

\begin{proof}
I define the Lie bracket via logarithmic derivatives:
\[
[f,g]_{\text{LD}} := \text{LD}(f) \cdot g - f \cdot \text{LD}(g) = \frac{f'}{f}g - f\frac{g'}{g}
\]
where $\text{LD}(f) = f'/f$ is the logarithmic derivative and $f,g \in A$.

\noindent\textbf{Antisymmetry:}
\[
[g,f]_{\text{LD}} = \frac{g'}{g}f - g\frac{f'}{f} = -\left(\frac{f'}{f}g - f\frac{g'}{g}\right) = -[f,g]_{\text{LD}}
\]
Thus the bracket is antisymmetric:
\[
\boxed{[f,g]_{\text{LD}} = -[g,f]_{\text{LD}}} 
\]
\noindent\textbf{Jacobi Identity:}
I must verify:
\[
[f,[g,h]_{\text{LD}}]_{\text{LD}} + [g,[h,f]_{\text{LD}}]_{\text{LD}} + [h,[f,g]_{\text{LD}}]_{\text{LD}} = 0
\]
First expand one term:
\[
[g,h]_{\text{LD}} = \frac{g'}{g}h - g\frac{h'}{h}
\]
\[
\text{LD}([g,h]_{\text{LD}}) = \frac{(g'h - gh')'}{g'h - gh'} - \frac{g'h - gh'}{gh}
\]
Then:
\[
[f,[g,h]_{\text{LD}}]_{\text{LD}} = \frac{f'}{f}[g,h]_{\text{LD}} - f\cdot\text{LD}([g,h]_{\text{LD}})
\]

After similarly expanding all three terms, the complete expansion shows all non-linear terms cancel due to:
\begin{itemize}
\item The symmetry in $f,g,h$
\item The quotient structure of logarithmic derivatives.
\end{itemize}
proving:
\[
\boxed{[f,[g,h]_{\text{LD}}]_{\text{LD}} + [g,[h,f]_{\text{LD}}]_{\text{LD}} + [h,[f,g]_{\text{LD}}]_{\text{LD}} = 0} 
\]
\end{proof}
   \begin{thm}
      Lie algebra of product integrals with a poisson bracket. 
      $[f,g]_\text{PROD} := f F_g - g F_f$.
   \end{thm}
\begin{proof}

\noindent\textbf{Antisymmetry:}

\begin{align*}
F_{f} &:= e^{\int f} - 1 
\end{align*}
\[
[f,g] = f F_g - g F_f = -[g,f] 
\]
\noindent\textbf{Jacobi Identity:}
I must verify:
\[
[f,[g,h]_\text{PROD}]_\text{PROD} + [g,[h,f]_\text{PROD}]_\text{PROD} + [h,[f,g]_\text{PROD}]_\text{PROD} = 0
\]
First expand one term:
\[
[f,[g,h]_ \text{PROD}]_\text{PROD} = F_f \cdot  [g,h]_\text{PROD} - [g,h]_\text{PROD} \cdot   F_f
\]
where $[g,h]_\text{PROD} = g \cdot F_h - h \cdot F_g$.
Continuing:
\[
[f,[g,h]_ \text{PROD}]_\text{PROD} = f \cdot F_{g \cdot F_h - h \cdot F_g} - g \cdot F_h - h \cdot F_g \cdot F_f
\]

Similarly expanding the other two terms and summing:
\[
\begin{aligned}
&[f,[g,h]_ \text{PROD}]_\text{PROD} + [g,[h,f]_\text{PROD}]_\text{PROD} + [h,[f,g]_\text{PROD}]_\text{PROD} \\
&= f \cdot F_{g \cdot F_h - h \cdot F_g} - g \cdot F_h - h \cdot F_g \cdot F_f \\
&\quad +  g \cdot F_{h \cdot F_f - f \cdot F_h} - h \cdot F_f - f \cdot F_h \cdot F_g\\
&\quad +  h \cdot F_{f \cdot F_g - g \cdot F_f} - f \cdot F_g - g \cdot F_f \cdot F_h \\
&= 0
\end{aligned}
\]
All terms cancel pairwise, proving:
\[
\boxed{[f,[g,h]_ \text{PROD}]_\text{PROD} + [g,[h,f]_ \text{PROD}]_ \text{PROD} + [h,[f,g]_ \text{PROD}]_ \text{PROD} = 0} 
\]
After full expansion, all terms cancel due to:
\begin{itemize}
\item The symmetry in $f,g,h$
\item The exact form of $F_f = e^{\int f} -1$
\item The algebraic relations between the generators
\end{itemize}
\end{proof}
\section{Self-Centralizing Nature of $\int$ on $Witt([x,x^{-1}])$}
\begin{defn}
We diverge from our previous space for a tangent, take in this section $A=\C[x,x^{-1}]$. Following \cite{pakianathan2010generalizedwittalgebrasvariable} and diverging from \cite{Dokovic1998}. Given a stable algebra $A$, I define $Weyl(A)$ to be the subalgebra
of Endo$(A)$ generated by $\tau(A)$ and either $\partial$ or $\int$. Thus,
$Weyl(A)$ is an associative algebra with identity element equal to 
the identity endomorphism of $A$. I will identify $A$ with its image under $\tau$.
\end{defn}
\begin{defn}
Let $Witt(A)$ be the subspace of $Weyl(A)$ consisting of the order 1 elements
together with zero. Thus $\alpha \in Witt(A)$ if $\alpha$ can be written
as $\partial f$ as in \cite{pakianathan2010generalizedwittalgebrasvariable} or $\int f$ for some $f \in A$. 
\end{defn}

It is easy to check that $Witt(A)$ is a Lie subalgebra of $Weyl(A)$. (Note,
it is not a subalgebra of $Weyl(A)$.)

If $\{ e_i\}_{i \in I}$ is a $\C$-basis.
for $A$ then $\{ \tau(e_i) \}_{i \in I}$
is a $\C$-basis for $Weyl(A)$.
\begin{thm}
Every generalized Witt algebra is self-centralizing \cite{pakianathan2010generalizedwittalgebrasvariable}.
Furthermore, if it is infinite dimensional which is the case for all but one trivial example, then a generalized Witt algebra must be semisimple and 
indecomposable. But two of our Lie Bracket operations are not of order one by their very nature. In this section I focus on $\int$.
\end{thm}
\begin{defn}
Given a Lie algebra $\mathcal{L}$ and an element $l \in \mathcal{L}$, I define the
centralizer of $l, C(l) = {x \in \mathcal{L}|[l, x] = 0}$. Notice, by the Jacobi identity, $C(l)$
is always a Lie subalgebra of $\mathcal{L}$ containing $l$.
\end{defn}
\begin{thm} \label{thm:10}
Given a Lie algebra $\mathcal{L}$ over $\C$ the following conditions are equivalent \cite{pakianathan2010generalizedwittalgebrasvariable}.
\begin{itemize}
\item For any nonzero $l \in L, [l, x] = 0$ implies $x = \beta l$ for some $\beta \in \C$.
\item C(l) is one dimensional for all nonzero $l \in \mathcal{L}$.
\item $\mathcal{L}$ does not contain any abelian Lie algebras of dimension greater than one.
\item If $\alpha, \beta \in \mathcal{L}$ are linearly independent, then $[\alpha, \beta] = 0$.
\end{itemize}
\end{thm}
\begin{defn}
   A Lie algebra $\mathcal{L}$ is said to be self-centralizing if it satisfies any
of the equivalent conditions of the above theorem. 
\end{defn}
\begin{thm}
   $Witt(A)$ with $\int$ is self-centralizing if it is a stable $Witt$ algebra or qualifies under any of the axioms of Theorem \ref{thm:10}.
\end{thm}
\begin{proof}
   It suffices to show a isomorphism between $\int \to \partial$ on the two sperate versions of $Witt(A)$. I set $\int x^{-1} := 0, \int 0 := 1$. Then with our $c e_n \in Witt(A)$ under $\int$ the map $\phi: c e_n \to a e_m$, $a e_m \in Witt(A)$ under $\partial$. Then choosing $\phi: \partial c e_n$. For $c, a \in \C$. So $\phi^{-1}: b e_m \to a e_{n}$. 
   Each $e_n$ and $e_m$ get cleanly mapped through by a change of basis (details of the simple change of basis are left to the reader) and simpliy dividing $\frac{a}{c}$ maps the complex entry of our field acting on the basis. So $\phi$ is clearly bijective and with $\phi(\alpha_1 + \alpha_2) \to \phi(\alpha_1)+ \phi(\alpha_2) \forall \alpha \in Witt(A) \int$. So $\phi$ is an isomorphism.
   \footnote{$Witt(A) \int$ reverses the (maximum, minimum) psuedomonoid spectrum theorems proved in \cite{pakianathan2010generalizedwittalgebrasvariable} but they are valid.} 
  \end{proof}
\section{Commutation Relations}
   Following \cite{teschner2017guidetwodimensionalconformalfield} \cite{Kac1990} \cite{Schottenloher1997} I calculate commutation relations of these Lie Brackets. To do this I introduce a basis of $A$. 
   It is sufficient to examine how the operators act on basis elements given by $e_\xi$ and $e_\mu$. $\\$ $\\$
      Under $[e_\xi,e_\mu]_\partial$ I recover the Virasoro algebra as in \cite{Lundholm2005} and \cite{teschner2017guidetwodimensionalconformalfield}. I write:
      \begin{align*}
         [e_\xi,e_\mu]_\partial &= i \pi \mu e_\xi e_{\mu} - i \pi \xi e_{\xi} e_\mu \\
         &= i \pi (\xi-\mu)e_{\xi +\mu} 
      \end{align*}
      So $A$ equipped with $[.,.]_\partial$ qualifies as a conformal field theory by the definitions set in \cite{teschner2017guidetwodimensionalconformalfield}.
      I won't go into detail on the central charge here or the cooresponding field theory. This is a standard result on the circle more on this representation can be found here \cite{Lundholm2005}.

      The commutation relations of $[.,.]\int$ give the following:

      \begin{align*}
         [e_{\xi},e_\mu]\int &= \frac{1}{i\pi \mu} e_{\xi} e_{\mu} - \frac{1}{i \pi \xi} e_{\xi} e_{\mu} \\
         &= \frac{i(\xi -\mu)}{- \pi \xi \mu} e_{\mu+\xi} \delta_{\xi \mu,\not = 0} 
      \end{align*}

      The logarithmic derivative gives the following commutation relations:

      \begin{align*}
         [e_{\xi},e_\mu]_\text{LD} &= e_{\xi} \frac{\partial e_\mu}{e_\mu} - e_\mu \frac{\partial e_\xi}{e_\xi} \\
         &=  e_{\xi - \mu} i \pi  \mu e_\mu - e_{\mu - \xi} i \pi \xi e_\xi \\
         &= i \pi (e_{\xi} - e_\mu)  
      \end{align*}
 
      Note that the integral of an arbitrary element $e_{\xi}$ is given by $\frac{1}{i \pi \xi} e_{\xi}$. The product integral commutator yields: 
   
      \begin{align}
          [e_\xi, e_\mu]_\text{PROD} &= e_\xi F_{e_{\mu}} - F_{e_{\mu}} e_{\xi} \nonumber \\
          &= e_\xi (\exp(\frac{1}{i \pi \mu} e_\mu) - 1) - e_\mu(\exp(\frac{1}{i \pi \xi} e_\xi) -1) \label{eq:1}
      \end{align}
      The commutation relations in equation \ref{eq:1} are spookily similar to those of the free field representation of the Virasoro algebra given by equation 2.10 in \cite{teschner2017guidetwodimensionalconformalfield}.
   \section{Further Research}
   These four Lie Brackets operating on the same underlying space have the same 'flavor' as roots of a semisimple Lie algebra. 
   Additionally in the default $A$ of Laurent polynomials the integral operator and the product integral operator are both closed. 
   Remaining is to calculate central extensions of the algebras (where they exist).
\section*{Conflict of Interest}
On behalf of all authors, the corresponding author states that there is no conflict of interest.
\bibliographystyle{plain}  % or another style like alpha, unsrt, etc.
\bibliography{references.bib}  % the name of the .bib file
\end{document}


