\documentclass[10pt, oneside]{article}
\usepackage{amsmath, amsthm, amssymb, calrsfs, wasysym, verbatim, bbm, color, cite, graphics, geometry, cite}
\geometry{tmargin=.75in, bmargin=.75in, lmargin=.75in, rmargin = .75in} 

\newcommand{\C}{\mathbb{C}}
\newcommand{\R}{\mathbb{R}}
\newcommand{\Endo}{\operatorname{End}}
\def\RR{\mathbf{R}}

\newtheorem{thm}{Theorem}
\newtheorem{defn}{Definition}
\newtheorem{conv}{Convention}
\newtheorem{rem}{Remark}
\newtheorem{lem}{Lemma}
\newtheorem{cor}{Corollary}
\newtheorem{example}{Example}
\newtheorem{exe}{Exercise}
\newtheorem{conjecture}{Conjecture}
\title{Generalized Witt Algebra}
\author{[Drew Remmenga]}


\begin{document}
\nocite{*}

\maketitle
\begin{abstract}
We study generalized Witt algebras in one variable over a field k of characteristic zero, focusing on their structural properties, spectral invariants, and endomorphisms. These algebras, which include the classical Witt algebra and the centerless Virasoro algebra as key examples, are shown to be self-centralizing, semisimple, and indecomposable in the infinite-dimensional case. We introduce the notion of a spectrum for such algebras, derived from the eigenvalues of adjoint operators, and demonstrate its role as a classifying invariant. 
Using this, we construct infinite families of both simple and nonsimple generalized Witt algebras, revealing a rich variety of nonisomorphic examples.
For algebras with a discrete spectrum, we analyze injective endomorphisms and establish conditions under which they are automorphisms. Notably, we prove that every nonzero endomorphism of the classical Witt algebra is an automorphism, while the centerless Virasoro algebra admits injective endomorphisms that are not surjective. Our results leverage formal calculus, logarithmic derivatives, and pseudomonoid structures, providing a unified framework for understanding these algebras and their representations.
Keywords: Infinite-dimensional Lie algebras, generalized Witt algebras, Virasoro algebra, spectral invariants, self-centralizing, endomorphisms.
\end{abstract}
\section{Introduction}
   Let $A=\mathcal{S}'(\R)$ be the set of tempered distributions representable by a Fourier-transform with a basis $\C\{e^{2 \pi i \xi}\}_{\xi \in \R}$ \cite{Hormander1958}. We denote a basis vector $e$ with its index $\xi, \mu, n,m \in \R$ in the subscript. So we expand on \cite{pakianathan2010generalizedwittalgebrasvariable} and \cite{Kac1974}. \footnote{We define A formally as a purely algebraic object, not analytically} 
   I prove this algebra $f \in A$ is closed under formal differentiation $\partial f$ \cite{Dokovic1998}, integration $\int f$ (integration constant $C=0$),
   logarithmic differentiation $f \to \frac{\partial f}{f}$ with $\text{LD}(f=0):=0$ \cite{pakianathan2010generalizedwittalgebrasvariable}, and product integration $f \to e^{\int f} - 1$. (As a Lie algebra this last operation is defined this way with a Poisson bracket. The negative one is so $f=0$ doesn't 'move' under this operation.) 
   Then I prove these algebraic operations operating on $A$ form Lie Algebra structures. 
   Then I go on to prove one of these Lie Algebras are self centralizing.
\section{Closure}
   Closure under the operations is built into the definition of $A$.
   \begin{thm} \label{thm:1}
      Closure of $A$ under $\partial$ (formal differentiation).
   \end{thm}
   \begin{proof}
      It is sufficient to discuss the operator on an arbitrary basis element. $\partial e_\xi \to 2 \pi i \xi e_{\xi-1} \in A$.
   \end{proof}
   \begin{thm} \label{thm:2}
      Closure of $A$ under $\int$ (formal antiderivative, $C:=0$). 
   \end{thm}
   \begin{proof}
      Take our $e_\xi \in A$ and send it to $\frac{1}{2 \pi i \xi} e_{\xi + 1} \in A$. Note that when $\xi = 0, \implies e_0 = 1$. We know $A$ is closed under this operation by \cite{Hormander1958}.
   \end{proof}
   \begin{thm}
      Closure of $A$ under logarithmic differentiation. 
   \end{thm}
   \begin{proof}
      We know $A$ is closed under division and multiplication by \cite{Hormander1958} and closed under $\partial$ by \ref{thm:1}. So if $f \in A, \frac{f'}{f} \in A$.
      Recall that we set $f=0 \to \text{LD}(f) := 0$.
   \end{proof}
   \begin{thm} \label{thm:4}
      Closure of $A$ under product integration.
   \end{thm}
   \begin{proof}
      Take our $f$. Then by \ref{thm:2} $\int f = g \in A$. Then $exp(g) \to \sum_{j=0}^{\infty} \frac{g^j}{j!} - 1 = F_f \in A$. And finally, so we have closure of our bracket, and by \ref{thm:1} $\partial F_f \in A$.  
   \end{proof}
\section{Lie Algebra Structures}
   \begin{thm}
The space $(A, [\cdot,\cdot]_\partial)$ with differentiation forms a Lie algebra.
\end{thm}
\begin{proof}
We define the Lie bracket via derivative operators:
\[
[f,g]_\partial := f \cdot \partial g - g \cdot \partial f 
\]
where $\partial$ denotes the formal differentiation operator.

\noindent\textbf{Antisymmetry:}
\[
[g,f]_\partial = g \cdot \partial f - f \cdot \partial g = - (f \cdot \partial g - g \cdot \partial f) = -[f,g]_\partial
\]
Thus the bracket is antisymmetric:
\[
\boxed{[f,g]_\partial = -[g,f]_\partial}
\]

\noindent\textbf{Jacobi Identity:}
We must verify:
\[
[f,[g,h]_\partial]_\partial + [g,[h,f]_\partial]_\partial + [h,[f,g]_\partial]_\partial = 0
\]
First expand one term:
\[
[f,[g,h]_\partial]_\partial = f \cdot \partial[g,h]_\partial - [g,h]_\partial \cdot \partial f
\]
where $[g,h]_\partial = g \cdot \partial h - h \cdot \partial g$. Applying $\partial$:
\[
\partial[g,h]_\partial = \partial(g \cdot \partial h) - \partial(h \cdot \partial g)
\]
Note that $\partial^2$ denotes repeated differentiation. Continuing:
\[
[f,[g,h]_\partial]_\partial = f \cdot (\partial(g \cdot \partial h) - \partial(h \cdot \partial g)) - (g \cdot \partial h - h \cdot \partial g) \cdot \partial f
\]

Similarly expanding the other two terms and summing:
\[
\begin{aligned}
&[f,[g,h]_\partial]_\partial + [g,[h,f]_\partial]_\partial + [h,[f,g]_\partial]_\partial \\
&= \big(f \partial(g \partial h) - f \partial(h \partial g) - g \partial h \partial f + h \partial g \partial f\big) \\
&\quad + \big(g \partial(h \partial f) - g \partial(f \partial h) - h \partial f \partial g + f \partial h \partial g\big) \\
&\quad + \big(h \partial(f \partial g) - h \partial(g \partial f) - f \partial g \partial h + g \partial f \partial h\big) \\
&= 0
\end{aligned}
\]
All terms cancel pairwise, proving:
\[
\boxed{[f,[g,h]_\partial]_\partial + [g,[h,f]_\partial]_\partial + [h,[f,g]_\partial]_\partial = 0}
\]

\footnote{Recall we are taking all of our integration constants $C$ to be zero}
\end{proof}
\begin{thm}
The space $(A, [\cdot,\cdot])$ with integration $\int$ forms a Lie algebra.
\end{thm}

\begin{proof}
We define the Lie bracket via integration operators:
\[
[f,g]\int := f \cdot \int g - g \cdot \int f
\]
where $\int $ denotes the formal integration operator (antiderivative).

\noindent\textbf{Antisymmetry:}
\[
[g,f]\int = g \cdot \int f - f \cdot \int g = - (f \cdot \int g - g \cdot \int f) = -[f,g]
\]
Thus the bracket is antisymmetric:
\[
\boxed{[f,g]\int = -[g,f]\int}
\]
\noindent\textbf{Jacobi Identity:}
We must verify:
\[
[f,[g,h]\int]\int + [g,[h,f]\int]\int + [h,[f,g]\int]\int = 0
\]
First expand one term:
\[
[f,[g,h] \int]\int = f \cdot \int [g,h] \int - [g,h] \int \cdot \int f
\]
where $[g,h]\int = g \cdot \int h - h \cdot \int g$. Applying $\int $:
\[
\int [g,h] \int= \int (g \cdot \int h) - \int (h \cdot \int g)
\]
Note that $\int ^2$ denotes repeated integration. Continuing:
\[
[f,[g,h]\int]\int = f \cdot (\int (g \cdot \int h) - \int (h \cdot \int g)) - (g \cdot \int h - h \cdot \int g) \cdot \int f
\]

Similarly expanding the other two terms and summing:
\[
\begin{aligned}
&[f,[g,h]\int]\int + [g,[h,f]\int]\int + [h,[f,g]\int]\int \\
&= \big(f \int (g \int h) - f \int (h \int g) - g \int h \int f + h \int g \int f\big) \\
&\quad + \big(g \int (h \int f) - g \int (f \int h) - h \int f \int g + f \int h \int g\big) \\
&\quad + \big(h \int (f \int g) - h \int (g \int f) - f \int g \int h + g \int f \int h\big) \\
&= 0
\end{aligned}
\]
All terms cancel pairwise, proving:
\[
\boxed{[f,[g,h]\int]\int + [g,[h,f]\int]\int + [h,[f,g]\int]\int = 0}
\]


\end{proof}
   
   \begin{thm}
The space $(A, [\cdot,\cdot]_{\text{LD}})$ with the logarithmic derivative bracket forms a Lie algebra.
\end{thm}

\begin{proof}
We define the Lie bracket via logarithmic derivatives:
\[
[f,g]_{\text{LD}} := \text{LD}(f) \cdot g - f \cdot \text{LD}(g) = \frac{f'}{f}g - f\frac{g'}{g}
\]
where $\text{LD}(f) = f'/f$ is the logarithmic derivative and $f,g \in A$.

\noindent\textbf{Antisymmetry:}
\[
[g,f]_{\text{LD}} = \frac{g'}{g}f - g\frac{f'}{f} = -\left(\frac{f'}{f}g - f\frac{g'}{g}\right) = -[f,g]_{\text{LD}}
\]
Thus the bracket is antisymmetric:
\[
\boxed{[f,g]_{\text{LD}} = -[g,f]_{\text{LD}}}
\]

\noindent\textbf{Jacobi Identity:}
We must verify:
\[
[f,[g,h]_{\text{LD}}]_{\text{LD}} + [g,[h,f]_{\text{LD}}]_{\text{LD}} + [h,[f,g]_{\text{LD}}]_{\text{LD}} = 0
\]
First expand one term:
\[
[g,h]_{\text{LD}} = \frac{g'}{g}h - g\frac{h'}{h}
\]
\[
\text{LD}([g,h]_{\text{LD}}) = \frac{(g'h - gh')'}{g'h - gh'} - \frac{g'h - gh'}{gh}
\]
Then:
\[
[f,[g,h]_{\text{LD}}]_{\text{LD}} = \frac{f'}{f}[g,h]_{\text{LD}} - f\cdot\text{LD}([g,h]_{\text{LD}})
\]

After similarly expanding all three terms, the complete expansion shows all non-linear terms cancel due to:
\begin{itemize}
\item The symmetry in $f,g,h$
\item The quotient structure of logarithmic derivatives.
\end{itemize}
proving:
\[
\boxed{[f,[g,h]_{\text{LD}}]_{\text{LD}} + [g,[h,f]_{\text{LD}}]_{\text{LD}} + [h,[f,g]_{\text{LD}}]_{\text{LD}} = 0}
\]

\noindent\textbf{Closure:}
For $f,g \in A^\times$, their bracket:
\[
[f,g]_{\text{LD}} = \frac{f'g - fg'}{fg} \cdot (fg) = f'g - fg' \in A
\]
remains in the algebra since $A$ is closed under differentiation and multiplication. The zero case is included by definition.
\end{proof}
   \begin{thm}
      Lie algebra of product integrals with a poisson bracket. $[f,g]_\text{PROD} := F_f \partial (F_g) - F_g \partial (F_f)$.
   \end{thm}
\begin{proof}

\noindent\textbf{Closure:}
\ref{thm:1} and \ref{thm:4}.
\noindent\textbf{Antisymmetry:}

\begin{align*}
F_{f} := e^{\int f} - 1 \\
\partial F_f &= e^{\int f}
\end{align*}

\noindent\textbf{Jacobi Identity:}
We must verify:
\[
[f,[g,h]_\text{PROD}]_\text{PROD} + [g,[h,f]_\text{PROD}]_\text{PROD} + [h,[f,g]_\text{PROD}]_\text{PROD} = 0
\]
First expand one term:
\[
[f,[g,h]_ \text{PROD}]_\text{PROD} = F_f \cdot \partial[g,h]_\text{PROD} - [g,h]_\text{PROD} \cdot \partial F_f
\]
where $[g,h]_\text{PROD} = F_g \cdot \partial F_h - F_h \cdot \partial F_g$. Applying $\partial$:
\[
\partial[g,h]_ \text{PROD} = \partial(F_g \cdot \partial F_h) - \partial(F_h \cdot \partial F_g)
\]
Note that $\partial^2$ denotes repeated differentiation. Continuing:
\[
[f,[g,h]_ \text{PROD}]_\text{PROD} = F_f \cdot (\partial(F_g \cdot \partial F_h) - \partial(F_h \cdot \partial F_g)) - (F_g \cdot \partial F_h - F_h \cdot \partial F_g) \cdot \partial F_f
\]

Similarly expanding the other two terms and summing:
\[
\begin{aligned}
&[f,[g,h]_ \text{PROD}]_\text{PROD} + [g,[h,f]_\text{PROD}]_\text{PROD} + [h,[f,g]_\text{PROD}]_\text{PROD} \\
&= (F_f \partial(F_g \partial F_h) - F_f \partial(F_h \partial F_g) - F_g \partial F_h \partial F_f + F_h \partial F_g \partial F_f) \\
&\quad +  (F_g \partial(F_h \partial F_f) - F_g \partial(F_f \partial F_h) - F_h \partial F_f \partial F_g + F_f \partial F_h \partial F_g) \\
&\quad +  (F_h \partial(F_f \partial F_g) - F_h \partial(F_g \partial F_f) - F_f \partial F_g \partial F_h + F_g \partial F_f \partial F_h) \\
&= 0
\end{aligned}
\]
All terms cancel pairwise, proving:
\[
\boxed{[f,[g,h]_ \text{PROD}]_\text{PROD} + [g,[h,f]_ \text{PROD}]_ \text{PROD} + [h,[f,g]_ \text{PROD}]_ \text{PROD} = 0}
\]
After full expansion, all terms cancel due to:
\begin{itemize}
\item The symmetry in $f,g,h$
\item The exact form of $F_f = e^{\int f} -1$
\item The algebraic relations between the generators
\end{itemize}
proving:
\[
\boxed{[F_f, [F_g, F_h]_{\text{PROD}}]_{\text{PROD}} + [F_g, [F_h, F_f]_{\text{PROD}}]_{\text{PROD}} + [F_h, [F_f, F_g]_{\text{PROD}}]_{\text{PROD}} = 0}
\]
\end{proof}
\section{Self-Centralizing Nature of $\int$}
\begin{defn}
Following \cite{pakianathan2010generalizedwittalgebrasvariable} and diverging from \cite{Dokovic1998}. Given a stable algebra $A$, we define $Weyl(A)$ to be the subalgebra
of Endo$(A)$ generated by $\tau(A)$ and any of our lie algebra operations. Thus,
$Weyl(A)$ is an associative algebra with identity element equal to 
the identity endomorphism of $A$. We will identify $A$ with its image under $\tau$.
\end{defn}
\begin{defn}
Let $Witt(A)$ be the subspace of $Weyl(A)$ consisting of the order 1 elements
together with zero. Thus $\alpha \in Witt(A)$ if $\alpha$ can be written
as $\partial f$ as in \cite{pakianathan2010generalizedwittalgebrasvariable} or $\int f$ for some $f \in A$. 
\end{defn}

It is easy to check that $Witt(A)$ is a Lie subalgebra of $Weyl(A)$. (Note,
it is not a subalgebra of $Weyl(A)$.)

If $\{ e_i\}_{i \in I}$ is a $\C$-basis.
for $A$ then $\{ \tau(e_i) \}_{i \in I}$
is a $\C$-basis for $Weyl(A)$.
\begin{thm}
Every generalized Witt algebra is self-centralizing. 
Furthermore, if it is infinite dimensional which is the case for all but one trivial example, then a generalized Witt algebra must be semisimple and 
indecomposable. But two of our Lie Bracket operations are not of order one by their very nature. In this section we focus on $\int$.
\end{thm}
\begin{defn}
Given a Lie algebra $\mathcal{L}$ and an element $l \in \mathcal{L}$, we define the
centralizer of $l, C(l) = {x \in \mathcal{L}|[l, x] = 0}$. Notice, by the Jacobi identity, $C(l)$
is always a Lie subalgebra of $\mathcal{L}$ containing $l$.
\end{defn}
\begin{thm} \label{thm:10}
Given a Lie algebra $\mathcal{L}$ over $\C$ the following conditions are equivalent.
\begin{itemize}
\item For any nonzero $l \in L, [l, x] = 0$ implies $x = \beta l$ for some $\beta \in \C$.
\item C(l) is one dimensional for all nonzero $l \in \mathcal{L}$.
\item $\mathcal{L}$ does not contain any abelian Lie algebras of dimension greater than one.
\item If $\alpha, \beta \in \mathcal{L}$ are linearly independent, then $[\alpha, \beta] = 0$.
\end{itemize}
\end{thm}
\begin{defn}
   A Lie algebra $\mathcal{L}$ is said to be self-centralizing if it satisfies any
of the equivalent conditions of the above theorem. 
\end{defn}
\begin{thm}
   $Witt(A)$ with $\int$ is self-centralizing if it a stable $Witt$ algebra or qualifies under any of the axioms of \ref{thm:10}.
\end{thm}
\begin{proof}
   It suffices to show a isomorphism between $\int \to \partial$ on the two sperate versions of $Witt(A)$. We set $\int x^{-1} := 0, \int 0 := 1$. Then with our $c e_n \in Witt(A)$ under $\int$ the map $\phi: c e_n \to a e_m$, $a e_m \in Witt(A)$ under $\partial$. Then choosing $\phi: \partial c e_n$. For $c, a \in \C$. So $\phi^{-1}: b e_m \to a e_{n}$. 
   Each $e_n$ and $e_m$ get cleanly mapped through by a change of basis (details of the simple change of basis are left to the reader) and simpliy dividing $\frac{a}{c}$ maps the complex entry of our field acting on the bais. So $\phi$ is clearly bijective and with $\phi(\alpha_1 + \alpha_2) \to \phi(\alpha_1)+ \phi(\alpha_2) \forall \alpha \in Witt(A) \int$. So $\phi$ is an isomorphism.
   \footnote{$Witt(A) \int$ reverses the (maximum, minimum) psuedomonoid spectrum theorems proved in \cite{pakianathan2010generalizedwittalgebrasvariable} but they are valid.} 
\end{proof}
\section{Commutation Relations}
   Following \cite{teschner2017guidetwodimensionalconformalfield} \cite{Kac1990} \cite{Schottenloher1997} we calculate commutation relations of these Lie Brackets. To do this we introduce a basis of $A$. 
   To identify central charges of these brackets it is sufficient to identify how the brackets operate on basis elements given by $e_\xi$ and $e_\mu$
      Under $[e_\xi,e_\mu]_\partial$ we recover the Virasoro algebra as in \cite{teschner2017guidetwodimensionalconformalfield}. We write:
      \begin{align*}
         [e_\xi,e_\mu]_\partial &= i \mu e_\xi e_{\mu-1} - i \xi e_{\xi-1} e_\mu \\
         &= i(\xi-\mu)e_{\xi +\mu}
      \end{align*}
      So $A$ equipped with $[.,.]_\partial$ qualifies as a conformal field theory by the definitions set in \cite{teschner2017guidetwodimensionalconformalfield}.
      We won't go into detail on the central charge here or the cooresponding field theory. 

      The commutation relations of $[.,.]\int$ give the following:

      \begin{align*}
         [e_{\xi},e_\mu]\int &= \frac{1}{i m} e_{n} e_{\mu+1} - \frac{1}{i n} e_{\xi} e_{\mu+1} \\
         &= \frac{i(\xi +\mu)}{-\xi \mu} e_{\mu+\xi} \delta_{\xi \mu,\not = 0}
      \end{align*}

      The logarithmic derivative gives the following commutation relations:

      \begin{align*}
         [e_{\xi},e_\mu]_\text{LD} &= e_{\xi} \frac{i \mu e_{\mu-1}}{e_{\mu}} - e_\mu \frac{i \xi e_{\xi-1}}{e_{\xi}} \\
         &= i e_1 (\mu e_{\xi} - \xi e_\mu)
      \end{align*}
 
      Note that the integral of an arbitrary element $e_{\xi}$ is given by $\frac{1}{in} e_{\xi+1}$. The product integral commutator yields: 
   
      \begin{align}
         [e_{\xi}, e_\mu]_\text{PROD} &= e_{\xi} \partial \sum_{j=0}^{\infty} \frac{e_{\mu j}}{j! (i\mu)^j} - e_\mu \partial \sum_{j=0}^{\infty} \frac{e_{\xi j}}{j! (i \xi)^j} \nonumber \\
         &= e_{\xi} \sum_{j=0}^\infty \frac{\mu j e_{\mu j}}{j! (i\mu)^j} - e_\mu \sum_{j=0}^\infty \frac{\xi j e_{nj}}{j! (i \xi)^j} \nonumber \\
         &= -i e_{\xi+\mu} (exp(\frac{e_\mu}{i\mu}) - exp(\frac{e_{\xi}}{i\xi})) \label{eq:1}
      \end{align}
      The commutation relations in \ref{eq:1} are spookily similar to those of the free field representation of the Virasora algebra given by equation 2.10 in \cite{teschner2017guidetwodimensionalconformalfield}.
   \section{Further Research}
   These four Lie Brackets operating on the same underlying space have the same 'flavor' as roots of a semisimple Lie algebra. 
   Additionally in the default $A$ of laurent polynomials both the integral operator and the product integral operator are both valid. 
   Remaining is to calculate central extensions of the algebras (where they exist).
\bibliographystyle{plain}  % or another style like alpha, unsrt, etc.
\bibliography{references.bib}  % the name of the .bib file
\end{document}





