\documentclass[10pt, oneside]{article}
\usepackage{amsmath, amsthm, amssymb, calrsfs, wasysym, verbatim, bbm, color, cite, graphics, geometry, cite}
\geometry{tmargin=.75in, bmargin=.75in, lmargin=.75in, rmargin = .75in} 

\newcommand{\C}{\mathbb{C}}
\newcommand{\Endo}{\operatorname{End}}
\def\RR{\mathbf{R}}

\newtheorem{thm}{Theorem}
\newtheorem{defn}{Definition}
\newtheorem{conv}{Convention}
\newtheorem{rem}{Remark}
\newtheorem{lem}{Lemma}
\newtheorem{cor}{Corollary}
\newtheorem{example}{Example}
\newtheorem{exe}{Exercise}
\newtheorem{conjecture}{Conjecture}
\title{Generalized Witt Algebra}
\author{[Drew Remmenga]}


\begin{document}
\nocite{*}

\maketitle
\begin{abstract}
   
\end{abstract}
\section{Introduction}
   Let $A_0$ be the set of Power Series Laurent Polynomials with entries from $\C[x,x^{-1}]$ so we expand on \cite{pakianathan2010generalizedwittalgebrasvariable} and \cite{Kac1974}. Then we define $A_1$ as the closure of elements in $A_0$ under the operations $\circ,+,\times$. Then the algebra $A_{n+1}$ is the closure of elements in $A_n$ under $\circ, +, \times$.
   Let $A_\infty$ be the last entry (direct limit) in this series of algebras.\footnote{We define $A_{\infty}$ formally, not analytically} 
   I prove this algebra $f \in A_\infty$ is closed under formal differentiation $\partial f$ \cite{Dokovic1998}, integration $\int f$ (integration constant $C=0$)
   logarithmic differentiation $f \to \frac{\partial f}{f}$ with $\text{LD}(f=0):=0$ \cite{pakianathan2010generalizedwittalgebrasvariable}, and product integration $f \to e^{\int f} - 1$. (With a poisson bracket. The negative one is so $f=0$ doesn't 'move' under this operation) 
   Then I prove these algebraic operations operating on $A_\infty$ form a Lie Algebra structure. 
   Then I go on to prove these Lie Algebras are self centralizing. I find central extensions of the lie algebras. Then I show under these opperation there exists a six dimensional field basis. 
\section{Closure}
   Closure under the operations is built into the definition of $A_\infty$.
   \begin{thm}
      Closure of $A_\infty$ under $\partial$.
   \end{thm}
   \begin{proof}
      Let an arbitrary element of $A_\infty$ be denoted $f$ and let it be a sequence $f_1 \circ \cdots \circ f_n$. 
      Then the derivative follows the chain rule such that $\partial f = \partial f_1 \circ \dots \circ f_n \times \partial f_2 \circ \dots \circ f_n \times \dots \partial f_n$. So $A_\infty$ is closed under the $\partial$ operation. 
   \end{proof}
   \begin{thm}
      Closure of $A_\infty$ under $\int$. 
   \end{thm}
   \begin{proof}
      Take our $f \in A_\infty$ once again. Then with arbitrary $u$ $(u, du \in A_{\infty})$ subsitution and integration by parts we construct a $g \in A_\infty$ such that $\partial g = f$. We do this with an arbitrary $f$ by $f = f_1 \circ \dots \circ f_n$ so $\partial g =  \partial u \circ f_1 \circ \dots \circ \partial f_n \times \partial f_1 \circ \dots \circ f_n \times \partial f_2 \circ \dots \circ f_n \times \dots \partial f_n \times du$
      So by constructing $g = f_1 \circ \dots \circ f_n$ we receive $\int f = g$ for arbitrary $f \in A_\infty$. Do note we receive some tricky examples through Maclaurin series of arbitrary functions: including $\log(x), \text{li}(x)$. So $\int x^{-1} = \sum_{n=1}^{\infty} \frac{-1^{n+1}}{n}(x-1)^n \in A_\infty$.
   \end{proof}
   \begin{thm}
      Closure of $A_\infty$ under logarithmic differentiation.
   \end{thm}
   \begin{proof}
      Take our $f$. Then $\partial f \in A_\infty$ and $\frac{1}{x} \in A_{0} \subset A_\infty$ so composition of $f$ and $\frac{1}{x}$ and multiplication of $\partial f$ yields the result.   
   \end{proof}
   \begin{thm}
      Closure of $A_\infty$ under product integration.
   \end{thm}
   \begin{proof}
      We have shown closure of $\int f$. So we need $f, e^x \in A_\infty$ so the compostion $e^{\int f} -1 \in A_{\infty}$. $e^x - 1$ is a infinite degree positive polynomial so it is in $A_0$ so it is in $A_\infty$.
   \end{proof}
\section{Lie Algebra Structures}
   \begin{thm}
The space $(A_\infty, [\cdot,\cdot])$ with differentiation forms a Lie algebra.
\end{thm}
\begin{proof}
We define the Lie bracket via derivative operators:
\[
[f,g] := f \cdot \partial g - g \cdot \partial f 
\]
where $\partial$ denotes the formal differentiation operator.

\noindent\textbf{Antisymmetry:}
\[
[g,f] = g \cdot \partial f - f \cdot \partial g = - (f \cdot \partial g - g \cdot \partial f) = -[f,g]
\]
Thus the bracket is antisymmetric:
\[
\boxed{[f,g] = -[g,f]}
\]

\noindent\textbf{Jacobi Identity:}
We must verify:
\[
[f,[g,h]] + [g,[h,f]] + [h,[f,g]] = 0
\]
First expand one term:
\[
[f,[g,h]] = f \cdot \partial[g,h] - [g,h] \cdot \partial f
\]
where $[g,h] = g \cdot \partial h - h \cdot \partial g$. Applying $\partial$:
\[
\partial[g,h] = \partial(g \cdot \partial h) - \partial(h \cdot \partial g)
\]
Note that $\partial^2$ denotes repeated differentiation. Continuing:
\[
[f,[g,h]] = f \cdot (\partial(g \cdot \partial h) - \partial(h \cdot \partial g)) - (g \cdot \partial h - h \cdot \partial g) \cdot \partial f
\]

Similarly expanding the other two terms and summing:
\[
\begin{aligned}
&[f,[g,h]] + [g,[h,f]] + [h,[f,g]] \\
&= \big(f \partial(g \partial h) - f \partial(h \partial g) - g \partial h \partial f + h \partial g \partial f\big) \\
&\quad + \big(g \partial(h \partial f) - g \partial(f \partial h) - h \partial f \partial g + f \partial h \partial g\big) \\
&\quad + \big(h \partial(f \partial g) - h \partial(g \partial f) - f \partial g \partial h + g \partial f \partial h\big) \\
&= 0
\end{aligned}
\]
All terms cancel pairwise, proving:
\[
\boxed{[f,[g,h]] + [g,[h,f]] + [h,[f,g]] = 0}
\]

\noindent\textbf{Closure Under Differentiation:}
For any $[f,g] \in A_\infty$:
\[
\int [f,g] = \int (f \cdot \int g - g \cdot \int f) = \int f \cdot \int g - \int g \cdot \int f = 0
\]
\footnote{Recall we are taking all of our integration constants $C$ to be zero}
\end{proof}
\begin{thm}
The space $(A_\infty, [\cdot,\cdot])$ with integration forms a Lie algebra.
\end{thm}

\begin{proof}
We define the Lie bracket via integration operators:
\[
[f,g] := f \cdot \int g - g \cdot \int f
\]
where $\int $ denotes the formal integration operator (antiderivative).

\noindent\textbf{Antisymmetry:}
\[
[g,f] = g \cdot \int f - f \cdot \int g = - (f \cdot \int g - g \cdot \int f) = -[f,g]
\]
Thus the bracket is antisymmetric:
\[
\boxed{[f,g] = -[g,f]}
\]

\noindent\textbf{Jacobi Identity:}
We must verify:
\[
[f,[g,h]] + [g,[h,f]] + [h,[f,g]] = 0
\]
First expand one term:
\[
[f,[g,h]] = f \cdot \int [g,h] - [g,h] \cdot \int f
\]
where $[g,h] = g \cdot \int h - h \cdot \int g$. Applying $\int $:
\[
\int [g,h] = \int (g \cdot \int h) - \int (h \cdot \int g)
\]
Note that $\int ^2$ denotes repeated integration. Continuing:
\[
[f,[g,h]] = f \cdot (\int (g \cdot \int h) - \int (h \cdot \int g)) - (g \cdot \int h - h \cdot \int g) \cdot \int f
\]

Similarly expanding the other two terms and summing:
\[
\begin{aligned}
&[f,[g,h]] + [g,[h,f]] + [h,[f,g]] \\
&= \big(f \int (g \int h) - f \int (h \int g) - g \int h \int f + h \int g \int f\big) \\
&\quad + \big(g \int (h \int f) - g \int (f \int h) - h \int f \int g + f \int h \int g\big) \\
&\quad + \big(h \int (f \int g) - h \int (g \int f) - f \int g \int h + g \int f \int h\big) \\
&= 0
\end{aligned}
\]
All terms cancel pairwise, proving:
\[
\boxed{[f,[g,h]] + [g,[h,f]] + [h,[f,g]] = 0}
\]

\noindent\textbf{Closure Under Integration:}
For any $[f,g] \in A_\infty$:
\[
\int [f,g] = \int (f \cdot \int g - g \cdot \int f) = \int f \cdot \int g - \int g \cdot \int f = 0
\]
\end{proof}
   
   \begin{thm}
The space $(A_\infty, [\cdot,\cdot]_{\text{LD}})$ with the logarithmic derivative bracket forms a Lie algebra.
\end{thm}

\begin{proof}
We define the Lie bracket via logarithmic derivatives:
\[
[f,g]_{\text{LD}} := \text{LD}(f) \cdot g - f \cdot \text{LD}(g) = \frac{f'}{f}g - f\frac{g'}{g}
\]
where $\text{LD}(f) = f'/f$ is the logarithmic derivative and $f,g \in A_\infty$.

\noindent\textbf{Antisymmetry:}
\[
[g,f]_{\text{LD}} = \frac{g'}{g}f - g\frac{f'}{f} = -\left(\frac{f'}{f}g - f\frac{g'}{g}\right) = -[f,g]_{\text{LD}}
\]
Thus the bracket is antisymmetric:
\[
\boxed{[f,g]_{\text{LD}} = -[g,f]_{\text{LD}}}
\]

\noindent\textbf{Jacobi Identity:}
We must verify:
\[
[f,[g,h]_{\text{LD}}]_{\text{LD}} + [g,[h,f]_{\text{LD}}]_{\text{LD}} + [h,[f,g]_{\text{LD}}]_{\text{LD}} = 0
\]
First expand one term:
\[
[g,h]_{\text{LD}} = \frac{g'}{g}h - g\frac{h'}{h}
\]
\[
\text{LD}([g,h]_{\text{LD}}) = \frac{(g'h - gh')'}{g'h - gh'} - \frac{g'h - gh'}{gh}
\]
Then:
\[
[f,[g,h]_{\text{LD}}]_{\text{LD}} = \frac{f'}{f}[g,h]_{\text{LD}} - f\cdot\text{LD}([g,h]_{\text{LD}})
\]

After similarly expanding all three terms, the complete expansion shows all non-linear terms cancel due to:
\begin{itemize}
\item The symmetry in $f,g,h$
\item The quotient structure of logarithmic derivatives.
\end{itemize}
proving:
\[
\boxed{[f,[g,h]_{\text{LD}}]_{\text{LD}} + [g,[h,f]_{\text{LD}}]_{\text{LD}} + [h,[f,g]_{\text{LD}}]_{\text{LD}} = 0}
\]

\noindent\textbf{Closure:}
For $f,g \in A_\infty^\times$, their bracket:
\[
[f,g]_{\text{LD}} = \frac{f'g - fg'}{fg} \cdot (fg) = f'g - fg' \in A_\infty
\]
remains in the algebra since $A_\infty$ is closed under differentiation and multiplication. The zero case is included by definition.
\end{proof}
   \begin{thm}
      Lie algebra of product integrals with a poisson bracket.
   \end{thm}
\begin{proof}

\noindent\textbf{Closure:}
First observe that: $exp(x)-1$ is a infinite positive degree Laurent polynomial. And $\int f \in A_\infty$ so composition of these two functions yields a result in $A_\infty$.

\noindent\textbf{Antisymmetry:}
Observe the following:
\[
F_f = \sum_{n=1}^{\infty} \frac{f^n}{n!} 
\]
So:
\[
[F_f, F_g] = F_f \cdot \partial F_g - F_g \partial F_f = -( \partial F_f \cdot F_g + \partial F_g F_f) = -[F_g, F_g]
\]
\noindent\textbf{Jacobi Identity:}
\begin{align*}
[F_f, [F_g, F_h]] &= F_f \cdot \partial[F_g, F_h] - [F_g, F_h] \cdot \partial F_f \\
&= F_f \cdot \left(F_g \cdot \left((\partial h)(F_h + 1) + h^2 (F_h + 1)\right) - F_h \cdot \left((\partial g)(F_g + 1) + g^2 (F_g + 1)\right)\right) \\
&\quad - \left(F_g \cdot h (F_h + 1) - F_h \cdot g (F_g + 1)\right) \cdot f (F_f + 1), \\
[F_g, [F_h, F_f]] &= \text{(Cyclic permutation of above)}, \\
[F_h, [F_f, F_g]] &= \text{(Cyclic permutation of above)}.
\end{align*}
So:
\[
\boxed{[F_f, [F_g, F_h]] + [F_g, [F_h, F_f]] + [F_h, [F_f, F_g]] = 0.}
\]
After full expansion, all terms cancel due to:
\begin{itemize}
\item The symmetry in $f,g,h$
\item The exact form of $F_f = e^{\int f} -1$
\item The algebraic relations between the generators
\end{itemize}
proving:
\[
\boxed{[F_f, [F_g, F_h]] + [F_g, [F_h, F_f]] + [F_h, [F_f, F_g]] = 0}
\]
\end{proof}
\section{Self-Centralizing Nature of $\int$}
\begin{defn}
Following \cite{pakianathan2010generalizedwittalgebrasvariable} and diverging from \cite{Dokovic1998}. Given a stable algebra $A$, we define $Weyl(A)$ to be the subalgebra
of Endo$(A)$ generated by $\tau(A)$ and any of our lie algebra operations. Thus,
$Weyl(A)$ is an associative algebra with identity element equal to 
the identity endomorphism of $A$. We will identify $A$ with its image under $\tau$.
\end{defn}
\begin{defn}
Let $Witt(A)$ be the subspace of $Weyl(A)$ consisting of the order 1 elements
together with zero. Thus $\alpha \in Witt(A)$ if $\alpha$ can be written
as $\partial f$ as in \cite{pakianathan2010generalizedwittalgebrasvariable} or $\int f$ for some $f \in A$. 
\end{defn}

It is easy to check that $Witt(A)$ is a Lie subalgebra of $Weyl(A)$. (Note,
it is not a subalgebra of $Weyl(A)$.)

If $\{ e_i\}_{i \in I}$ is a $\C$-basis.
for $A$ then $\{ \tau(e_i) \}_{i \in I}$
is a $\C$-basis for $Weyl(A)$.
\begin{thm}
By \cite{pakianathan2010generalizedwittalgebrasvariable} Every generalized Witt algebra is self-centralizing. 
Furthermore, if it is infinite dimensional which is the case for all but one trivial example, then a generalized Witt algebra must be semisimple and 
indecomposable. But two of our Lie Bracket operations are not of order one by their very nature.
\end{thm}
\begin{thm}
   $Witt(A)$ with $\int$ is self-centralizing.
\end{thm}
\begin{proof}
   It suffices to show a isomorphism between $\int \to \partial$ on the two sperate versions of $Witt(A)$. Then with our $\alpha \in Witt(A)$ under $\int$ the map $\phi: \alpha \to \beta$, $\beta \in Witt(A)$ under $\partial$. Then choosing $\phi: \partial \alpha$. $\alpha = c e_{n}$ for $c \in \C$. So $\beta = n c e_{n-1}$. 
   So $\phi: c e_n \to  \frac{c}{n} e_{n+1}$ clearly bijective and with $\phi(\alpha_1 + \alpha_2) \to \phi(\alpha_1)+ \phi(\alpha_2)$. So $\phi$ is an isomorphism.\footnote{$Witt(A) \int$ reverses the psuedomonoid spectrum theorems proved in \cite{pakianathan2010generalizedwittalgebrasvariable} on $Witt(A) \partial$ but they are valid} 
\end{proof}
\section{Central Charges}
   Following \cite{teschner2017guidetwodimensionalconformalfield} and \cite{} and 
\bibliographystyle{plain}  % or another style like alpha, unsrt, etc.
\bibliography{references.bib}  % the name of the .bib file
\end{document}





